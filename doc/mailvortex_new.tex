\documentclass[10pt,a4paper,appendixprefix,twocolumn,draft]{scrbook}

\usepackage{svn}
\SVN $Revision: 295 $
\SVN $Author: mgwerder $
\SVN $Date: 2016-06-14 18:30:29 +0200 (Di, 14 Jun 2016) $
\SVN $URL: https://qetesh.gwerder.net/svn/mailvortex/doc/mailvortex.tex $
\SVN $Id: mailvortex.tex 295 2016-06-14 16:30:29Z mgwerder $

\author{Martin Gwerder (06-073-787)}
\date{\SVNDate}

% enable graphics inclusion
\usepackage[final]{graphicx}

\usepackage{pdfsync} 

\usepackage{MnSymbol} % required for the arrow

%\usepackage{underscore}

% titlepage geometry change
\usepackage[paper=a4paper,top=2cm,bottom=2cm,inner=2cm,outer=2cm]{geometry}% http://ctan.org/pkg/geometry

\usepackage{scrhack}

\usepackage[autocite=superscript,
            backref=true,
            backend=bibtex,
            %bibstyle=AnonMail,
            hyperref=true,
            url=true,
            isbn=true,
            maxcitenames=3,
            maxbibnames=100,
            block=none,
            sorting=anyt]{biblatex}

%\bibstyle{alphadin}
\addbibresource{mailvortex}
\addbibresource{inc/bib/unclassified/Anonbib/anonbib}
\usepackage{csquotes}

% For Multipage listing in appendix
\usepackage[final]{listings}
\usepackage{caption}
\usepackage[framemethod=tikz]{mdframed}
\usepackage[many]{tcolorbox}
\tcbuselibrary{listings}

% enable raggedright in tables
\usepackage{array}
% for diagonal divided cells in tables
\usepackage{makecell}
\renewcommand\theadfont{\bfseries}

% footnotes for tables
\usepackage{tablefootnote}
%\makesavenoteenv{tabular}

% enable placement of floating images
\usepackage{float}

%enable page spanning tables
\usepackage{supertabular}

%enable hypelinks
\usepackage[pdftex,pdfusetitle]{hyperref}
\hypersetup{
	hidelinks,
  pdfpagelayout=TwoPageRight,
%	bookmarks=true,
  pdfstartview=Fit
}

% Link above tables and figures
%\usepackage[hypcap]{caption}

% support repetitive footnotes
\usepackage{fixfoot}

% Document annotations
\usepackage[nomargin]{fixme}
\fxuselayouts{pdfnote}

%enable attached files
\usepackage{attachfile}

%enable word separation
\usepackage[english]{babel}

% enable nice references
\usepackage{fancyref}

% enable superscript  for 1st, 2nd, 3rd etc.
\usepackage[super]{nth}

% Enable indexes
\usepackage{makeidx}
\makeindex 

% format appendix
\usepackage[titletoc,page,title]{appendix} %enable appendix 

% set numbering for subsubsections
\usepackage{tocstyle}
\setcounter{secnumdepth}{3}
\setcounter{tocdepth}{4}

% Sans serif font for the whole document
\renewcommand{\familydefault}{\sfdefault}
\usepackage[T1]{fontenc}
\usepackage{times}

\usepackage{amsmath}

% No paragraph indentation
\setlength\parindent{0pt} 
\setlength\parskip{6pt} 

% For coments and similar
\usepackage{verbatim}

% For Page background color
\usepackage{afterpage}
\usepackage{xcolor} 

% Required for definitions environment
\usepackage{hanging}
\usepackage{ragged2e}
\newenvironment{entry}{\par\leavevmode\hangpara{1.5mm}{1}\ignorespaces}{\RaggedRight\par}
\newcommand*{\mainentry}[2]{{\bfseries{#1\label{def:#1}}}~#2\par}
\newcommand*{\subentry}[2]{\par~\begin{tabular}{p{\textwidth-.6cm}@{}}{\bfseries{\itshape{#1\label{def:#1}}}}~#2\end{tabular}}

\newcommand*{\defref}[1]{\hyperref[def:#1]{#1}}
\title{MessageVortex}
\subtitle{Transport Independent Messaging anonymous to \nth{3} Parties}
\author{Martin Gwerder (06-073-787)}
\date{\SVNDate}

\hypersetup{pdfinfo={
Subject={Privacy when using common internet transport protocols},
Keywords={Email; SMTP; MIME; S/MIME; POP3; IMAPv4; MailVortex; Anonymity}
}}

\lstset{ %
	backgroundcolor=\color{lightgray},   
	language=java,
	frame=single,
	numbers=left,
	numbersep=5pt,
	numberstyle=\tiny,
	basicstyle=\tiny
}

\lstdefinelanguage{ASN1}
{
	morekeywords=[1]{DEFINITIONS,AUTOMATIC,TAGS,BEGIN,END,%
		SEQUENCE,OF,CHOICE,ENUMERATED,NULL,SIZE,OPTIONAL,%
		OCTET,BIT,STRING,INTEGER,REAL,BOOLEAN,WITH,COMPONENTS},%
	commentstyle=\itshape,%
	morecomment=[l]{--},%
	basicstyle=\tiny\sffamily,
}

\lstdefinestyle{BashInputStyle}{
	language={},
	basicstyle=\tiny\sffamily,
	numbers=none,
	frame=tb,
	breaklines=true,
	prebreak={\mbox{\quad$\rhookswarrow$}},
	columns=fullflexible,
	backgroundcolor=\color{gray!20},
%	linewidth=0.9\linewidth,
%	xleftmargin=0.1\linewidth
}

\begin{document}
\newgeometry{margin=0.1cm, footskip=1mm}

\frontmatter

% include title page
\begin{titlepage}
\pagecolor{orange}\afterpage{\nopagecolor}
\begin{center}
\includegraphics[height=0.4\textwidth]{./inc/logo}~\\[1cm]

\textsc{\LARGE University of ???}\\[1.5cm]

\textsc{\Large PhD Thesis}\\[0.5cm]

% Title
\newcommand{\HRule}{\rule{\linewidth}{0.5mm}}
\HRule \\[0.4cm]
{ \huge \bfseries \makeatletter\@title\makeatother \\[0.4cm] }

\HRule \\[1.5cm]

% Author and supervisor
\begin{minipage}{0.6\textwidth}
\begin{flushleft} \large
\emph{Author:}\\
 \makeatletter\@author\makeatother
\end{flushleft}
\end{minipage}
\begin{minipage}{0.6\textwidth}
\begin{flushright} \large
\emph{Supervisor:} \\
 unknown
\end{flushright}
\end{minipage}

\vfill

% Bottom of the page
{\large \today}

\end{center}
\end{titlepage}



\restoregeometry

\begin{comment}
\begin{abstracts}        %this creates the heading for the abstract page
FIXME abstract is still missing
\end{abstracts}

\begin{acknowledgements}      %this creates the heading for the acknowlegments
I would like to thank my wife Cornelia and my lovely three kids (Saphira, Florian and Aurelius) for their patience and their support. Without them I could never have done this work.\par

FIXME Professor\par
FIXME university\par
FIXME family\par

I would like to acknowledge the thousands of individuals who have coded for the LaTeX project for free. It is due to their efforts that we can generate professionally typeset PDFs now.
\end{acknowledgements}
\end{comment}

\cleardoublepage
\makeatletter\renewcommand{\l@subsubsection}{\@dottedtocline{3}{7.4em}{4.5em}}\makeatother
\tableofcontents
\listoftables
\listoffigures

\mainmatter

%!TeX program=pdflatex
%!TeX encoding=utf8
%!TeX spellcheck = en_GB
%!TeX root = mailvortex_new.tex

\chapter{Introduction}
Numerous events in present and past have shown that data is broadly collected in the internet. Wether this is a problem or not may be a disputable fact. Undisputed is however that if data is not handled with care peoples are accused with numerous ``facts'' that are more than questionable. To show that this may happen even under complete democratic control we might refer to events such as the ``secret files scandal?? (or ``Fichenskandal'') in Switzerland. In the years  from 1900 to 1990 Swiss government collected 900'000 files in a secret archive (covering roughly 10\% of the natural and juristic entities within Switzerland at that time). 

A series of similar attempts to attack privacy on a global scale have been discovered by whistle blower Edward Snowden. The documents leaked in 2009 by him claim that there was a data collection starting in 2010. Since these documents are not publicly available it is hard to to prove claims based on these documents. However -- due to the fact that the documents were screened by a significant number of journalists spanning multiple countries, the information seems credible.

According to these documents (verified by \href{http://www.nrc.nl/nieuws/2013/11/23/nederland-sinds-1946-doelwit-van-nsa}{NRC}) NSA infiltrated more than 50k computers with malware to collect classified information. They furthermore infiltrated Telecom-Operators (executed by british GCHQ) such as Belgacom to collect data and targeted high member of governments even in associated states (such as the germans president mobile phone). A later published shortened list of ``selectors'' showed 68 telephone and fax numbers targeting enconomy, finance and agricultural parts of the german government.

This list of events shows that big players are collecting and storing vast amounts of data for future use. The list of events shows also that the use of this data has in the past been at least partially questionable. As a part of possible counter measures this work analyses the possibility of using state of the art technology to minimize the information footprint of a person on the internet. 

We leave a vast information footprint in our daily communication. On a regular email we disclose everything in an ``postcard'' to any entity on its way. Even when encrypting a message perfectly with todays technology (S/MIME\cite{RFC2045} or PGP\cite{RFC2015}) leaves at least the originating and the receiving entity disclosed. Most likely other relevant information such as ``message subject'', ``frequency of exchanged messages'', ``size of messages'', or ``client beeing used''. A good anonymity protocol has therefore far more attributes to cover than the message itself. Furthermore a protocol anonymising messages should not rely on the trust of infrastructure other than the infrastructure under control of the sending or receiving entity. 

Any central infrastructure is bound to be of special interest to anyone gathering data concerning the using entities of such a protocol. So central infrastructure has to be avoided.

In this work a new protocol is designed to allow message transfer through existing communication channels. These messages should be unobservable to any third party. This unobservability does not only cover the message itself but all metadata associated with it. 

\chapter{Main Research Question}
The main topic of this thesis was defined as follows:
\begin{itemize}
	\item Is it possible to have specialized asynchronous messaging protocol based on ``state of the science'' technologies offering a high level of unlikability (sender and receiver anonymity) towards an advisor with a high budget and privileged access to infrastructure?
\end{itemize}

Based on this main question there are several sub questions grouped around various topics:

\begin{itemize}
	\item What technologies and methods can be used to provide anonymity against a potential adverser?
	\item How can entities utilizing these technologies and methods be attacked?
	\item How can attacks targeting anonymity of a sending or receiving entity be mitigated by design?
\end{itemize}

\part{Methodes}
In this part of the dissertation we collect definitions, methods, and existing research relevant to the topic of this thesis

\chapter{Existing Research and Implementations on the Topic}
\section{Anonymity}
\DeclareFixedFootnote{\omitted}{footnotes omitted in quote}
As Anonymity we take the definition as specified in \cite{anon_terminology}.
\begin{quote}
	Anonymity of a subject means that the subject is not identifiable within a set of subjects, the anonymity set.\omitted
\end{quote}
and
\begin{quote}
	Anonymity of a subject from an attacker's perspective means that the attacker cannot sufficiently identify the subject within a set of subjects, the anonymity set.\omitted
\end{quote}

Whereas the anonymity set is defined as the set of all possible subjects.

Especially the second quote is very important to this paper. 
\subsection{$k$-Anonymity}
$k$-anonymity is a term introduced in \cite{k-anonymous:ccs2003}. This work claims that no one might be held responsible for an action if the action itself can only be identified as an action which has been taken by one unidentifiable entity out of $k$ entities.

The Document distinguishes between \textit{Sender $k$-anonymity} where the sending entity can only be narrowed down to a set of $k$ entities and \textit{Receiver $k$-anonymity} 

\subsection{$\ell$-Diversity}
In \cite{machanavajjhala2007diversity} an extended model of $k$-anonymity. According to the authors it is possible to break a $k$-anonymity set if there is additional Information available which may be merged into a data set so that a special entity can be filtered from the $k$-anonymity set. In other words if an anonymity set is to tightly specified a single additional background information might be sufficient to identify a specific entity in an anonymity set.

Their approach is to introduce an amount of invisible diversity into $k$-anonymous sets so that simple background knowledge is no longer sufficient to isolate a single member.

\subsection{$t$-Closeness}
While $\ell$-diversity protects the identity of an entity it does not prevent informatino gain. A subject which is in a class has the same attributes. This is where $t$-closeness\cite{li2007t} comes into play. $t$-closeness is defined as follows:

\begin{quote}
  An equivalence class is said to have $t$-closeness if the distance between the distribution of a sensitive attribute in this class and the distribution of the attribute in the whole table is no more than a threshold t. A table is said to have t-closeness if all equivalence classes have t-closeness.
\end{quote}

\section{Zero Trust}
Zero trust is not a truly researched model in systems engineering. It is however widely adopted. 

We refer in this work to the zero trust model when denying the trust in any infrastructure not directly controlled by the sending or receiving entity. This distrust extends especially but not exclusively to the network transporting the message, the nodes storing and forwarding messages, the backup taken from any system, and software, hardware and operators of all systems not explicitly trusted.

\section{Pseuidonymity}
As Pseudonymity we take the definition as specified in \cite{anon_terminology}.

\begin{quote}
	A pseudonym is an identifier of a subject other than one of the subject's real
	names. The subject which the pseudonym refers to is the holder of the pseudonym. A subject is pseudonymous if a pseudonym is used as identifier instead of one of its real names.\omitted
\end{quote}

\section{undetectability}
\section{unobservability}

\subsection{Ephemeral Identity}
\section{Single Use Reply Blocks}
\section{Censorship}
\subsection{Censorship Resistant}
\subsection{Parrot Coircumvention}
\subsection{Censorship Circumvention}
\subsubsection{Covert Channel}
\subsubsection{Spread Spectrum}

\section{Cryptography}
\subsection{Symmetric Encryption}
\subsubsection{RSA}
\subsubsection{Elliptic Curve Cryptogaphy}
\subsection{Asymmetric Encryption}

\section{System Implementations}
The following sections describe 
\subsection{Concepts}
\subsubsection{DC Networks}
\subsubsection{MIX Networks}
\subsubsection{Onion Routing}
\subsubsection{Remailer}

\subsection{Implementations}
The following sections emphasize on implementations of anonymising (and related) protocols regardless of their usage in the domain of messaging. It si a list of system classes or their specific implementations together with a short analysis of strength and weaknesses. Wherever possible we try to refer to original sources.
\subsubsection{Pseudonymous Remailer}
\subsubsection{Babel}
\subsubsection{Cypherpunk-Remailer}
\subsubsection{Mixmaster-Remailer}
\subsubsection{Mixminion-Remailer}
\subsubsection{Crowds}
\subsubsection{Herbivore}
\subsubsection{Dissent}
\subsubsection{P5}
\subsubsection{Gnutella}
\subsubsection{Gnutella2}
\subsubsection{Freenet}
\subsubsection{Darknet}
\subsubsection{Sneakernet}
\subsubsection{Hordes}
\subsubsection{Salsa}
\subsubsection{Hydra-Onion}

\section{Known Attacks}
In the following sections we emphasize on possible attacks to an anonymity preserving protocols. In the following sections we describe classes of attacks. These attacks may be used to attack the anonymity of any entity involved in the message channel. In a later stage we test the protocol for immunity against these classes of attacks.

\subsection{Broken Encryption Algorithms}
\subsection{Attacks Targeting Anonymity}
\subsubsection{Hotspot Attacks}
\subsubsection{Message Tagging and Tracing}
\subsubsection{Side Channel Attacks}

\subsubsection{Bugging Attacks}
\fxfatal{Bugging thru certificate/identity lookup}
\fxfatal{Bugging thru CRL lookup}
\fxfatal{Bugging thru DNS traffic}

\subsection{Denial of Service Attacks}
\subsubsection{Censorship}
\subsubsection{Credibility Attack}

\chapter{Applied Methodes}

\part{Results}

\chapter{MessageVortex - Transport Independent Messaging anonymous to \nth{3} Parties}
\chapter{Security Analysis}
\chapter{Additional Considerations}
\section{Storage of Messages}
The storage of messages sent though MessageVortex should be handled with great care. It seems on the first sight a good idea to merge all messages in a globally available storage such as the mail account of the receiving entity. However -- In doing so we would discover the message content to the providing party of a mail account. Since we handled the message with great care and tremendous costs up until this point it would be careless doing so. 

Storing them in a localized and receiving entity controlled storage is definitely a good idea but leaves security considerations like a backup possibly to an end user. This might be better but in effect a questionable decision. There is however a third option. By leaving the message unhandled on the last entity of the MessageVortex chain we may safely backup the data without disclosing the message content. Merging the content then dynamically through a specialized proxy would allow the user tu have a unified view on his without compromising the security.

\fxfatal{implemented in prototype?}

\part{Discussion}

\chapter{Anonymity}


\backmatter
\appendix
\part{Appendix}
\makeatletter
\g@addto@macro\appendix{%
	\renewcommand*{\chapterformat}{%
		{\chapapp\nobreakspace\thechapter\autodot\enskip}%
	}
	\renewcommand*{\chaptermarkformat}{%
		{\chapapp\nobreakspace\thechapter\autodot\enskip}%
	}
	\let\oldaddcontentsline\addcontentsline
	\newcommand\hackedaddcontentsline[3]{\oldaddcontentsline{#1}{#2}{\chapapp\nobreakspace#3}}
	\let\oldchapter\chapter
	\renewcommand*\chapter[1]{%
		\let\addcontentsline\hackedaddcontentsline%
		\oldchapter{#1}%
		\let\addcontentsline\oldaddcontentsline%
	}
}
\makeatother
\onecolumn
\chapter{ASN.1 representation of the protocol\label{app:asnone}}
\lstinputlisting[language=ASN1, caption={ASN.1 representation of the protocol}, backgroundcolor=\color{gray!10}]{../app/asn.1/messageBlocks.asn} 
\twocolumn


\chapter{Glossary}

\begin{entry}
  \mainentry{adverser}{FIXME}
\end{entry}

\begin{entry}
  \mainentry{Agent}{FIXME}
\end{entry}

\begin{entry}
  \mainentry{EWS}{FIXME}
\end{entry}

\begin{entry}
  \mainentry{IMAP}{IMAP (currently IMAPv4) is a typical protocol to be used between a \defref{Client MRA} and a \defref{Remote MDA}. It has been specified in its current version in \cite{RFC3501}. The protocol is capable of fully maintaining a server based message store. This includes the capability of adding, modifying and deleting messages and folders of a mailstore. It does not include however sening mails to other destinations outside the server based store.}
\end{entry}

\begin{entry}
	\mainentry{Item of Interest (IoI)}{FIXME}
\end{entry}

\begin{entry}
  \mainentry{LMTP}{FIXME}
\end{entry}

\begin{entry}
  \mainentry{Local Mail Store}{A Local Mail Store offers a persistent store on a local non volatile memory in which messages are beeing stored. A store may be flat or structured (eg. supports folders). A Local Mail Store may be an authoritative store for mails or a ``Cache Only'' copy. It is typically not a queue.}
\end{entry}

\begin{entry}
  \mainentry{mail server admin}{FIXME}
\end{entry}

\begin{entry}
  \mainentry{MDA}{An MDA provides an uniform access to a \defref{Local Mail Store}.}
  \subentry{Remote MDA}{A Remote MDA is typically supporting a specific access protocol to access the data stored within a \defref{Local Mail Store} .}
  \subentry{Local MDA}{A Local MDA is typically giving local applications access to a server store. This may be done thru an API, a named socket or similar mechanisms.}
\end{entry}

\begin{entry}
  \mainentry{MRA}{A Mail receiving Agent. This agent receives mails from a agent. Depending on the used protocol two subtypes of MRAs are available.}
  \subentry{Client MRA}{A client MRA picks up mails in the server mail storage from a remote MDA. Client MRAs usually connect thru a standard protocol which was designed for client access. Examples for such protocols are \defref{POP} or \defref{IMAP}}
  \subentry{Server MRA}{Unlike a Client MRA a server MRA listens passively for incomming connections and forwardes received Messages to a MTA for delivery and routing. A typical protocol supported by an Server MRA is \defref{SMTP}}
\end{entry}

\begin{entry}
  \mainentry{MS-OXMAPIHTTP}{FIXME}
\end{entry}

\begin{entry}
  \mainentry{MSA}{A Mail Sending Agent. This agent sends mails to a \defref{Server MRA}. }
\end{entry}

\begin{entry}
  \mainentry{MTA}{A Mail Transfer Agent. This transfer agent routes mails between other components. Typically  an MTA receives mails from an MRA and forwardes them to a MDA or MSA. The main task of a MTA is to provide reliable queues and solid track of all mails as long as they are not forwarded to another MTA or local storage.}
\end{entry}

\begin{entry}
  \mainentry{MTS}{A Mail Transfer Service. This is a set of agents which provide the functionallity tor send and receive Messages and forward them to a local or remote store.}
\end{entry}

\begin{entry}
  \mainentry{MSS}{A Mail Storage Service. This is a set of agents providing a reliable store for local mail accounts. It also provides Interfacing which enables clients to access the users mail.}
\end{entry}

\begin{entry}
  \mainentry{MUA}{A Mail User Agent. This user agent reads mails from a local storage and allows a user to read existing mails, create and modify mails.}
\end{entry}

\begin{entry}
  \mainentry{Privacy}{From the Oxford English Dictionary: ``
    \begin{enumerate}
      \item The state or condition of beeing withdrawn from the society of others, or from the public intrest; seclusion. The state or condition of beeing alone, undisturbed, or free from public attention, as a matter of choice or right; freedom from interference or intrusion.
      \item Private or retired place; private apartments; places of retreat.
      \item Absence or avoidance of publicity or display; a condition approaching to secrecy or concealment. Keeping of a secret.
      \item A private matter, a secret; private or personal matters or relations; The private parts.
      \item Intimacy, confidential relations.
      \item The state of being privy to some act.
    \end{enumerate}''\cite[FIXME]{OXFORD}\\
    In this work privacy is related to definition two. Mails should be able to be handled as a virtual private place where no one knows who is talking to whom and about what or how frequent (except for directly involved people).
  }
\end{entry}

\begin{entry}
  \mainentry{POP}{POP (currently in version 3) is a typical protocol to be used between a \defref{Client MRA} and a \defref{Remote MDA}. Unlike \defref{IMAP} it is not able to maintain a mail store. Its sole purpose is to fetch and delete mails in a server based store. Modifying Mails or even handling a complex folder structure is not doable with POP}
\end{entry}

\begin{entry}
  \mainentry{Service}{FIXME}
\end{entry}

\begin{entry}
  \mainentry{SMTP}{SMTP is the most commonly used protocol for sending mails across the internet. In its current version it has been specified in \cite{RFC5321}.}
\end{entry}

\begin{entry}
  \mainentry{Storage}{A store to keep data. It is assumed to be temporary or persistent in its nature.}
\end{entry}

\begin{entry}
  \mainentry{user}{FIXME}
\end{entry}

\begin{entry}
  \mainentry{UBE}{FIXME}
\end{entry}

\chapter{Bibliography}
{
  \renewcommand*{\bibfont}{\small}
  \printbibliography[title={},heading=none]
}

% additional reference entries
\index{Mail transport|see {Message Transport}}

% add the index
\printindex

\begin{comment}
% just a trick to make TexNicCenter Bibliography working
\bibliography{mailvortex}
\bibliography{inc/bib/unclassified/Anonbib/anonbib}
\end{comment}



\begin{comment}
% Some Notes 
http://www.rfc-editor.org/pubprocess.html
RFC2223 Instructions to RFC Authors
RFC2119 BCP14 Key words for use in RFCs to Indicate Requirement Levels
RFC3979 BCP79 Intellectual Property Rights in IETF Technology
RFC5378 BCP78 Rights Contributors Provide to the IETF Trust


http://tex.stackexchange.com/questions/36307/formatting-back-references-in-bibliography
http://www.cs.columbia.edu/irt/software/l2x/ l2x -- conversion from LaTeX to other formats Version 1.13
http://ftp.gwdg.de/pub/ctan/support/l2x/
http://tools.ietf.org/tools/xml2rfc2

http://www.zisc.ethz.ch/events/2003-2011/ISC2006Slides/FederrathZISCTalk.pdf

Professorliste
Dr. Christoph Sprenger (Part I)
-Prof. David Basin
Gregory Demay
Peter Gazi
Dr. Srdjan Marinovic
Dr. Sasa Radomirovic
Dr. Ralf Sasse

T. Hoefler
A. Perrig 
-Dr. Jan Camenisch (Keine Berechtigung)

-Srdjan Capkun (Keine Kapazität)
-David Basin  (Keine Kapazität)
\end{comment}
\end{document}
