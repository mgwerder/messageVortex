% !TeX spellcheck = en_US
\documentclass[]{article}
\usepackage{rfc}
\usepackage[autocite=superscript,
backref=true,
backend=bibtex,
hyperref=true,
url=true,
isbn=true,
maxcitenames=3,
maxbibnames=100,
block=none,
sorting=anyt]{biblatex}

\addbibresource{../messageVortex}
\addbibresource{../inc/bib/unclassified/Anonbib/anonbib}

% \rfc          - used when generating RFC instead of I-D
% \rfcnum{xxxx}
\i-d
\title{MessageVortex Protocol}
\author{Martin Gwerder}
\address{University of Basel, Switzerland}
\pubdate{Feb, 2018}


\begin{document}

\maketitle


\section*{Status of This Memo}
This memo defines an Experimental Protocol for the Internet community.  This memo does not specify an Internet standard of any kind.  Discussion and suggestions for improvement are requested. Distribution of this memo is unlimited.


\section*{Copyright Notice}
Copyright (C) The Internet Society (2018).

\section*{Abstract}
MessageVortex Protocol specifies messages that may be embedded into existing message transfer protocols such as SMTP or XMPP in order to send them anonymously from peer to peer. The protocol outperforms other protocols by completely decoupling transport from the final transmitter and receiver party. There is no trust  put into any infrastructure except for the infrastructure of the sending and receiving party of a message. Message flow is entirely selected by the creator of the routing block. Participating routing nodes gain no knowledge about messages even when collaborating. While third party anonymity may be always and one out of sender, and receiver anonymity may be achieved.

\newpage
\tableofcontents

\section{Introduction}

\section{Terminology}
The key words "MUST", "MUST NOT", "REQUIRED", "SHALL", "SHALL NOT", "SHOULD", "SHOULD NOT", "RECOMMENDED", "MAY", and "OPTIONAL" in this document are to be interpreted as described in \cite{RFC2119}.

\section{Archtectural Overview}

\subsection{Protocol Overview}

\section{Routing Layer}

\section{Blending Layer}

\section{Accounting Layer}

\section{References}

\section{Extensibility}

\section{Security Cosiderations}

\section{IANA Considerations}

\section{References}

	%\def\filespliter#1{\expandafter\intfilespliter#1\relax}
	%\def\intfilespliter#1 #2 #3\relax{ First: (#1), Second: (#2), Third: (#3) }
	%\DeclareFieldFormat{file}{\StrGobbleLeft{#1}{1}[\wtcGwM]\StrGobbleRight{\wtcGwM}{4}[\filename] \IfFileExists{\filename}{\attachfile{\filename}}{FIXME missing document link}}
	%\renewbibmacro{finentry}{\finentry\addspace \printfield{file}}
	%\renewcommand*{\bibfont}{\small}
	\printbibliography[title={},heading=none]%


\section*{Authors Adresses}

Martin Gwerder\\
Untere Parkstrasse 9\\
CH-5212 Hausen AG\\
Switzerland\\
~\\
Phone: +41 56 202 76.81\\
EMail: rfc@gwerder.net

\end{document}