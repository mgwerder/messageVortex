\documentclass[11pt,a4paper]{book}

\usepackage{svn}
\SVN $Revision$
\SVN $Author$
\SVN $Date$
\SVN $URL$
\SVN $Id$


\usepackage[backref=true,
            backend=bibtex,
            hyperref=true,
            url=true,
            isbn=true,
            maxcitenames=3,
            maxbibnames=100,
            block=none,
            sorting=anyt]{biblatex}
\addbibresource{AnonMail}
\usepackage{csquotes}

% enagle graphics inclusion
\usepackage{graphicx}

%enable page spanning tables
\usepackage{supertabular}

%enable hypelinks
\usepackage[pdfusetitle]{hyperref}

%enable attached files
\usepackage{attachfile}

%enable word separation
\usepackage[english]{babel}

% enable nice references
\usepackage{fancyref}

% Enable indexes
\usepackage{makeidx}
\makeindex 
\usepackage[titletoc]{appendix} %enable appendix

% set numbering for subsubsections
\setcounter{secnumdepth}{3}
\setcounter{tocdepth}{4}

% Sans serif font for the whole document
\renewcommand{\familydefault}{\sfdefault}
\usepackage[T1]{fontenc}
\usepackage{lmodern}

% No paragraph indentation
\setlength\parindent{0pt} 
\setlength\parskip{6pt} 

% For coments and similar
\usepackage{verbatim}

% For Page background color
\usepackage{afterpage}
\usepackage{xcolor} 

% Required for definitions environment
\usepackage{hanging}
\usepackage{ragged2e}
\newenvironment{entry}{\par\leavevmode\hangpara{1.5mm}{1}\ignorespaces}{\RaggedRight\par}
\newcommand*{\mainentry}[1]{{\bfseries{#1}}~}

\title{Possible solutions to implement email transfer offering anonymity towards third parties}
\author{Martin Gwerder}
\date{\SVNDate}

\hypersetup{pdfinfo={
Subject={Privacy when using mail},
Keywords={Email; SMTP; MIME; S/MIME; POP3; IMAPv4}
}}

\begin{document}
\frontmatter

% include title page
\begin{titlepage}
\pagecolor{orange}\afterpage{\nopagecolor}
\begin{center}
\includegraphics[height=0.4\textwidth]{./inc/logo}~\\[1cm]

\textsc{\LARGE University of ???}\\[1.5cm]

\textsc{\Large PhD Thesis}\\[0.5cm]

% Title
\newcommand{\HRule}{\rule{\linewidth}{0.5mm}}
\HRule \\[0.4cm]
{ \huge \bfseries \makeatletter\@title\makeatother \\[0.4cm] }

\HRule \\[1.5cm]

% Author and supervisor
\begin{minipage}{0.6\textwidth}
\begin{flushleft} \large
\emph{Author:}\\
 \makeatletter\@author\makeatother
\end{flushleft}
\end{minipage}
\begin{minipage}{0.6\textwidth}
\begin{flushright} \large
\emph{Supervisor:} \\
 unknown
\end{flushright}
\end{minipage}

\vfill

% Bottom of the page
{\large \today}

\end{center}
\end{titlepage}



\begin{comment}
\begin{abstracts}        %this creates the heading for the abstract page
FIXME abstract is still missing
\end{abstracts}

\begin{acknowledgements}      %this creates the heading for the acknowlegments
I would like to thank my wife Cornelia and my lovely three kids (Saphira, Florian and Aurelius) for their patience and their support. Without them I could never have done this work.\par

FIXME Professor\par
FIXME university\par
I would like to acknowledge the thousands of individuals who have coded for the LaTeX project for free. It is due to their efforts that we can generate professionally typeset PDFs now.
\end{acknowledgements}
\end{comment}


\tableofcontents
\listoftables
\listoffigures

\mainmatter

\chapter{Introduction}
This document describes a solution, which should offer anonymity against third parties when sending emails based on SMTP and the respective client protocols (e.g. IMAPv4 or POP3). This seemed to bother very few peoples up until information in Echelon became public due to an investigation by a commitee of the european parlament in 2001\cite{ECHELON}. Things settled again with local peaks up until a whistle blower named Edward Snowden disclosed $200000$ documents proofing activities of the NSA and other secret services. This led to the ``2013 mass surveillance disclosures'' and damaged the reputation of the american nation in many countries\cite{wiki:edward_snowden}.

\section{Overview over the current situation}
SMTP as defined in RFC5321\cite{RFC5321} is as of today (2013)  state of the art transmission protocol for electronic mail. It is standardized in its current version since 2008 and is one of the few protocols, which is marked as "Standard". While the protocol delivers reliable mail transfer between two endpoint (mail servers) the anonymity of the message content towards any mail server is not given (For a detailed analysis see \ref{chap:analysis}).\par

Anonymity against third party is not given due to the following facts.

\begin{itemize}
  \item There is not always an encryption available between a mail user agent (MUA) and the outgoing mail server.
	\item There is no way to guarantee that a mail transfer between two SMTP hosts is encrypted.
	\item There is no always an encryption available between a SMTP host and the MUA of the recipient.
  \item Encryption based on top level protocols (such as S/MIME or PGP) do hide the message content. The sender, recipient, the subject and some technical information (eg. MIME-Headers) are always in plain available and not protected as such.
	\item Even if there is a reliable encryption between all endpoints and none of the intermediate servers are compromised sender and recipients might still be identified thru traffic analysis.
\end{itemize}

Keeping the message content confidential is more and more relevant in these days. The more the importance of mail transfer in today's economy is growing the more is confidentiality and reliability a topic. Unfortunately Secret Services have already discovered the significance of today's mail traffic and start to analyse those. With the presence of Secret Services in the internet, actively investigating data the importance of a reliable data channel for today's messages has become increasingly important. 

Quick wins such as the use of "Onion Router Networks" (such as TOR) do not offer any additional security since the message content would be revealed in full to an eventual exit node and any mail server on its way to the recipient.

\section{Problem statement}
This work is an approach to extend the existing mail routing based on SMTP by an intermediate layer, which should offer anonymity against third party.\par

This work delivers the following results:

\begin{itemize}
  \item A throughout analysis of current technology and its weaknesses.\\
	      Although the Simple Mail Transfer Protocol (SMTP) is a well-implemented and well proven technology its weaknesses are well known. The SMTP protocol was originally defined in RFC821\cite{RFC821} by Johnathan B. Postel. At this time internet was only available to universities, some mayor companies and governments. The objective of Simple Mail Transfer Protocol (SMTP) is to transfer mail reliably and efficiently\cite[p.~1]{RFC821}. Confidentiality or having a tamper proof protocol was no design goal. Over the years many standards arose trying to close some of the gaps. Some of them are being used but most of them are not very common. 
	\item An analysis of possible approaches to improve the current standards.\\
	      Many standards and technologies do exist these days addressing parts of the issues mentioned above. A throughout research should be carried out to identify how can these technologies be combined to achieve the subsequent goals. Furthermore technology advanced. Namely in the field of cryptology few possibilities and ideas arose (such as new encryption classes [eg. elliptic curves] or the idea of crypto puzzles). Another field of research which emerged in the analysis of traffic flow is handled under the term ``Big Data'' where not single events but the sum of events is handled.
	\item A RFC document describing an approach offering a significant quality improvement of the existing solutions, which could be accepted by the internet community.
	\item A prototype reflecting at least the minimum baseline of the RFC document to reflect prove its functionality.\\
	      A prototype should be offered to show the feasibility. The Prototype should be a reference implementation and offer a quick way to use the new technology. It should be distributed under the LGPL license to simplify distribution of the technology. 
\end{itemize}

\section{Contributions}
This thesis contributes to the topic in the following senses:
\begin{itemize}
\item It introduces a consistent model for message delivery which includes all endpoints
\item It shows an approach based on existing protocols for anonymous communication
\end{itemize}

\chapter{Ground theory}
\section{Mail Transport}
\section{Anonymity}
\section{Identification and data signage}
\section{Encryption}
\section{Mix cascades}
\section{Remailers}
Agents which do accept Mails from one party and forward it to another party while modifying its content well known under the name of ``Remailers''. Wikipedia \cite{wiki:remailer} lists four types of Remailers.\par

Pseudonymous Remailers (or Type-0-Remailers) are remailers that establish a pseudonymity. This means that the senders Email-Address is removed and replaced by a pseudonymous E-Mailadress under the remailers control. This sender address may be used as an ordinary email-Adress to reeach the original sender of the mail. These types of Remailers allow to send mails while one or both recipients do not know their counterpart. The message (or at least parts of it) might be encrypted but do not have to be. For someone controling the Remailer it will always be possible to make a link between the pseudonymous mail address and a original mailadress. So pseudonymity is only granted towards people not controlling the remailer. Furthermore a person or organisation might be able to discover the Information tuple of Sender and pseudonymous email by analyzing messages and their timely context. So this remailer system is suspectible for traffic analysis.\par

Cypherpunk-Remailers (or Type-1-Remailers) do function a bit different. They take an encrypted message which was encrypted using the public key of the server, decrypt it and send it to a recipient. The original senders identity gets lost. A reply to a cypherpunk message is not possible. Messages sent to a cypherpunk server might contain messages to other cypherpunk remailers. This daisy-chaining of cypherpunk-nodes allows hiding the original sender-receiver-tuple from a single node. The first node knows only the the originating sender while the last node knows only the final recipient. All intermediate notes do only know the nodes they were linking. However if having traffic information of the entry and exit nodes the tuple might be discovered by traffic analysis.\par

Mixmaster remailer (or type-2-remailer) FIXME

Mixminion remailer (or type-3-remailer) FIXME


\subsection{Ethics of the Internet}
\cite[p.~1]{RFC1087}

\subsection{Possible legal issues}
One of the first questions I have been asked when working for this topic was: Is this legal? The question is important but not easy at all. The mail system is a global spanning network coming across almost any country of the world. Some of these countries consider almost any kind of secret as illegal as long as the country itself is not able to capture it. Some countries consider it as perfectly legal and some will generally accept its presence as long as the country or establishment is not endangered due to its usage. \par

My personal unscientific point is: I do not care. In my country it is definitely legal as long as I am a well behaved citizen (as long as I do not missuse this system to plan or do illegal actions). There are already proprietary systems available which offer the same functionality. All I do is adding this functionality to the common system instead of reinventing the wheel. There are however many very good reasons to have such a system. Correspondence about my health, my business relations, my friends or my family (to give just a couple of examples) should be kept private even in an open world. The misuse of information would cause tremendous damage and several events in time (which have been mentioned earlier) showed that there are many secret services and other players using any kind of information to achieve their own goals or the goals of associates. They do this regardless of any country borders or regulations. Since I have no means of controlling the flow of messages in the internet or the hubs where a mail is running thru I consider it as fair to generate an addon to compensate the lack of control in the existing system. Exactly as a car -- the system may be legal or illegal and it depends on the users whether he wants to use it or not and in what way.\par

To some extend the system is related to a remailer system. One of the most famous pseudonymous remailers. Wikipedia 



\chapter{Current situation}
As of today the de facto standard for asynchronous mail transfer is SMTP as defined in RFC5321\cite{RFC5321} and its predecessors. While the transfer protocol SMTP is quite compact, the protocol is enhanced with several standards for encryption, multimedia support and similar. A mail client offers today various support for a lot of sub-protocols. The following list is an excerpt of related sub-protocols which are either related to transport, reliability, identification or encryption. 

\section{Implemented protocols}
\subsection{SMTP \index{SMTP|textbf}}
The SMTP protocol is currently specified in \cite{RFC5321}. It specifies a method to deliver reliably asynchronous messages thru a specific transport medium (most of the time the internet). The Protocol makes a dstinction between a mail envelope and its content. The envelope contains the routing information which is the sender and the recipient. The content again is split into two parts. These parts are the headers (which do contain meta information about the message such as subject, reply address or a comprehensive list of all recipients) and the body which contains the message itself.\par

It furthermore introduces a simplistic model for mail communication. A more comprehensive model is introduced in the section \nameref{sec:mailTransport}. As the proposed model is not sufficient for a comprehensive end-to-end analysis.\par



FIXME incomplete section

\subsubsection{Mail transport\index{Message!Transport|textbf}}
\cite{RFC1870}
FIXME incomplete section

\subsubsection{encryption}
Encryption is the only anonymizing technology which is available. There are serveral kind of encryptions which have to be differentiated. Link encryption controls the E-Mail connection a guarantees that the whole communication between two servers is encrypted. It does however not guarantee that the message and routing information is protected all the way thru the network. Message encryption is a weaker encryption which is done at a higher level of the protocol stack. It guarantees that a message is end to end encrypted but discloses all routing and header information.\par

One kind of Mail link encryption is specified in \cite{RFC3207}. This RFC specifies that when a STARTTLS-Command is issued a TLS handshake initiating a encrypted link should be carried out between two Servers. Only not public servers (not published in DNS using MX records) may enforce the use of TLS. All public servers must allow non-TLS transport. Authentication thru this port is possible but usually not done. The STARTTLS specification states clearly that securing a link provides no end-to-end security. An attack to this mechanism is very simple. The only thing required is injecting a 454 error code when the client issues a STARTTLS. According to the document the sending server may then refuse to deliver the document but in reality this never happens in public SMTP servers.\par

For encryption between a mail endpoint (repective its MSA)  and the server MRA Clients may choose to use alternate ports which enforce a TLS handshake at the TCP handshake. This invalidates the possibility to disturb a connection while still in plain text modes with fake errorcodes but since it is a weak security anyway it makes really a difference. According to the \cite{RFC3207} document the port 587 should be used. On some servers the same functionality is provided on port 465. This was originally intended for mail transmission between two MSAs. The usage of this port has however never been standardized, violates \cite{RFC6409} and the port has been assigned to the URD Protocol by IANA.\par

The second type of encryption is message encryption. Message encryption does not cover the whole server communication starting from a specific point. It does only cover some parts or the full message body. The Two main protocols in use are S/MIME (As specified in \cite{RFC2311}) and PGP/MIME (as specified in \cite {RFC2015}; bases on \cite{RFC1847}). Both do reveal vital information to all involved parties of the mail transport and a possible third party observer thus completely invalidating anonymity. Informations which are visible to anyone are: 

\begin{itemize}
\item sender address (may be forged)
\item sender client (may be forged)
\item Recipient address
\item message subject 
\item the full routing path including all rewrites, timing information and intermediate hops.
\item the content type
\item Mime-Version
\item Date and time of sending
\end{itemize}

Any client or intermediate Server may furthermore add additional information of any kind (such as virus scanning information, anti spam taxation, reply address).

FIXME unfinished section

\subsection{MIME}
\cite{RFC2045}
\cite{RFC2046}
\cite{RFC2047}
\cite{RFC2048}
\cite{RFC2049}
FIXME incomplete section

\subsubsection{S/MIME}
\cite{RFC3851}
FIXME incomplete section

\subsubsection{PGP/MIME}
\cite{RFC2440}
FIXME incomplete section

\subsection{DNS}
\cite{RFC2929}
FIXME incomplete section

\subsubsection{DNSSEC}
\cite{RFC5155}
FIXME incomplete section

\subsubsection{Sender Policy Framework}
\cite{RFC4408}
\cite{RFC6652}
FIXME incomplete section

\subsubsection{Sender ID}
\cite{RFC4401}
FIXME incomplete section

\subsection{Transport Protocols}
FIXME incomplete section

\subsubsection{IPv4}
\cite{RFC760}
\cite{RFC791}
\cite{RFC1180}
\cite{RFC792}
\cite{RFC919}
\cite{RFC922}
\cite{RFC950}
\cite[p.~3]{RFC793}
FIXME incomplete section

\subsubsection{IPv6}
\cite{RFC2460}
FIXME incomplete section

\subsubsection{TCP}
FIXME incomplete section

\subsection{Remote MDA protocols}
FIXME incomplete section

\subsubsection{POP3}
\cite{RFC1939}
FIXME incomplete section

\subsubsection{IMAP}
\cite{RFC3501}
FIXME incomplete section



\chapter{Analysis of current situation\label{chap:analysis}}
\section{Current state of common Technology}
\subsection{Mailrouting}
\subsubsection{SMTP}
\subsubsection{LMTP}
\subsubsection{IMAP}
\subsubsection{POP3}
\subsubsection{MS-OXMAPIHTTP}
\section{Current state of available Technology}
\section{Missing Gap}
\section{Skeleton of Mails and mail transfer}




\chapter{Designing an approach}
\section{Defining system boundaries}
\subsection{Thread model}
As an adverser we assume the following attributes:
\begin{itemize}
\item Available founding is huge.
\item Can have own mailer infrastructure.
\item Is able to read, write or modify network data freely at any point of the net.
\end{itemize}
His intensions are:
\begin{itemize}
\item Discover message flows
\item Discover message contents
\item Identify users of the system
\end{itemize}

\subsection{User model}
The assumed user of the system is:
\begin{itemize}
\item Does care about privacy.
\item Has no special computer knowhow.
\item Has the ability to install a program or plugin.
\item Has no cryptographic knowhow.
\item Is using a device with enough calculation power to solve cryptographic tasks.
\end{itemize}
His intensions are:
\begin{itemize}
\item Send personal or confidential Information securely to another user
\end{itemize}
His expectations are:
\begin{itemize}
\item System should be easy to configure and maintain (in an ideal world: Zero touch). 
\item System should be fast.
\item System should be reliable.
\item System should work on any client he is using.
\item System should not be a legal problem to him or any of his peers.
\end{itemize}

\subsection{Mail server admin model}
The assumed mail server admin of the system is:
\begin{itemize}
\item Does care about privacy.
\item Has considerable computer knowhow.
\item Has the ability to install a program or plugin.
\item Has possibly no cryptographic knowhow.
\item Does know his own mail infrastructure.
\item Is using a device with enough calculation power to solve cryptographic tasks.
\end{itemize}
His intensions are:
\begin{itemize}
\item Support his users in sending personal or confidential information securely to another user
\end{itemize}
His expectations are:
\begin{itemize}
\item System should be easy to configure and maintain (in an ideal world: Zero touch). 
\item System should be fast.
\item System should be reliable.
\item System should work on any client he is using.
\item System should not be a legal problem for him or his company.
\item System should still allow him to do regulatory tasks such as virus scanning or backup.
\end{itemize}

\section{Basic Requirements of an approach}

\chapter{Specifying a target solution}
\section{Blocks}
\subsection{Preamble}
\begin{itemize}
  \item identity
  \item Request Number (RNR)
  \item Identity signature
  \item Message key
  \item Shuffle
\end{itemize}

\subsection{Routing block}
\begin{itemize}
  \item Next hop address
  \item RNR (must match preamble)
  \item Validity time
  \item delivery expiry
  \item Message checksum (optional)
  \item Padding  (optional)
\end{itemize}

\subsection{Address request block}

\section{Messages}
\subsection{Basecom}
\begin{itemize}
  \item Request Server keys
  \item Request identity (identity and key tuple)
  \item Request size quota (size in bytes and end time)
  \item Request Puzzle spec
  \item Request capabilities
\end{itemize}


\chapter{Verification of solution}
\section{User acceptance of the target system}
From a perspective of a user 
Collected requirements to a mail system:\par
\begin{center}
\tablefirsthead{\hline%
\multicolumn{1}{|l}{Requirement} &
\multicolumn{1}{|l}{cliteria} &
\multicolumn{1}{|l|}{Weight}\\\hline}
\tablehead{\hline%
\multicolumn{3}{|l|}{\small\sl continued from previous page}\\\hline
\multicolumn{1}{|l}{Requirement} &
\multicolumn{1}{|l}{cliteria} &
\multicolumn{1}{|l|}{Weight}\\\hline}
\tabletail{\hline
\multicolumn{3}{|l|}{\small\sl continued on next page}\\\hline}
\tablelasttail{\hline}
\bottomcaption{User acceptance requirements}
\begin{supertabular}{|p{5cm}|p{5cm}|c|}\hline
The System should be able to transport mails fast under normal conditions & Mails should travel with at least 1MB/min & 5\\\hline
The System should transport mails reliable & Mails should always arrive or their status should be retrievable & 9\\\hline
The System should offer anonymity against spying from third parties & Neither original sender nor final destination or any part of the message content should be determinable by any part of the system except for the original sender and the final recipient.& 9\\\hline
The system must be easy to handle &&8\\\hline
The system must be easy to install & Installation should be almost a ``single-click''-Thing. Details should be copied or accessed from the existing configurations.\\\hline
Messages should be prepared fast &&8\\\hline
The degree of anonymity should be controllable &  & 8\\\hline
\end{supertabular}
\end{center}

\section{Admin acceptance of the target system}
Collected requirements to a mail system from an admin perspective:\par
\begin{tabular}{|p{5cm}|p{5cm}|c|}\hline
Requirement& Criteria& weight\\\hline
The cost of mail routing should be controllable & under heavy load or due to resource shortage the system should be able to increase the cost for a sender & 5\\
The system should not be missusable for UCE & UCE mails should be too costly to send & 9\\
\hline\end{tabular}
\section{Possible attacks to the system}
\subsection{Generic DoS attacks}
It is always possible to overload a system. However due to the combination with cryptopuzzles it is very hard for an attacker to use costly system ressources (such as cpu for decrypting or encrypting messages) without having far higher resource costs on his side. FIXME
\subsubsection{Overloading single nodes}
The cost for detecting an illic message are very small (just two or three cypher blocks) while the costs for generating load are very high. FIXME
\subsection{Attacks on the users anonymity}
\subsection{Reputaional attacks}
\subsubsection{Misuse for sending spam}
\subsubsection{Misuse for covering illegal actions}




\backmatter
\begin{appendices}
\chapter{Definitions}
\begin{entry}\mainentry{Local MTA}{A local Mail Transfer Agent. This transfer agent reads mails to be sent to a remote account from a local storage and sends it to a MSA.}\end{entry}
\begin{entry}\mainentry{MRA}{A local Mail receiving Agent. This agent pulls either mails from a local MDA. An MDA might offer the possibility for a push pull delivery.}\end{entry}
\begin{entry}\mainentry{MSA}{A local Mail Sending Agent. This agent accepts mails to be sent to a remote MTA. }\end{entry}
\begin{entry}\mainentry{MUA}{A Mail User Agent. This user agent reads mails from a local storage and allows a user to read existing mails, create and modify mails.}\end{entry}
\begin{entry}\mainentry{Privacy}{From the Oxford English Dictionary: ``
  \begin{enumerate}
    \item The state or condition of beeing withdrawn from the society of others, or from the public intrest; seclusion. The state or condition of beeing alone, undisturbed, or free from public attention, as a matter of choice or right; freedom from interference or intrusion.
    \item Private or retired place; private apartments; places of retreat.
    \item Absence or avoidance of publicity or display; a condition approaching to secrecy or concealment. Keeping of a secret.
    \item A private matter, a secret; private or personal matters or relations; The private parts.
    \item Intimacy, confidential relations.
    \item The state of being privy to some act.
  \end{enumerate}''\cite[FIXME]{OXFORD}\par
  
  In this work privacy is related to definition two. Mails should be able to be handled as a virtual private place where no one knows who is talking to whom and about what or how frequent (except for directly involved people).
}\end{entry}


\chapter{Bibliography}
\printbibliography[title={},heading=none]
\begin{comment}
\bibliography{AnonMail}
\end{comment}
\end{appendices}

\begin{comment}
http://www.rfc-editor.org/pubprocess.html
RFC2223 Instructions to RFC Authors
RFC2119 BCP14 Key words for use in RFCs to Indicate Requirement Levels
RFC3979 BCP79 Intellectual Property Rights in IETF Technology
RFC5378 BCP78 Rights Contributors Provide to the IETF Trust


http://tex.stackexchange.com/questions/36307/formatting-back-references-in-bibliography
http://www.cs.columbia.edu/irt/software/l2x/ l2x -- conversion from LaTeX to other formats Version 1.13
http://ftp.gwdg.de/pub/ctan/support/l2x/
http://tools.ietf.org/tools/xml2rfc2

http://www.zisc.ethz.ch/events/2003-2011/ISC2006Slides/FederrathZISCTalk.pdf

Professorliste
Dr. Christoph Sprenger (Part I)
-Prof. David Basin
Gregory Demay
Peter Gazi
Dr. Srdjan Marinovic
Dr. Sasa Radomirovic
Dr. Ralf Sasse

T. Hoefler
A. Perrig 
-Dr. Jan Camenisch (Keine Berechtigung)

-Srdjan Capkun (Keine Kapazität)
-David Basin  (Keine Kapazität)
\end{comment}
\end{document}
