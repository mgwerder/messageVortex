%!TeX program=pdflatex
%!TeX encoding=utf8
%!TeX spellcheck = en_US
%!TeX root = ../../messageVortex.tex


% ********************************************************************************************************
% *** Decisions and Research
% ********************************************************************************************************
\part{The  MessageVortex System}
In the following section we describe the core functionalities 
\fxwarning{complete section}

MessageVortex is a protocol piggybacking common transport protocols somehow similar as S/MIME\cite{RFC2015} or PGP\cite{PGP} which are common transport protocols such as SMTP. 
\fxwarning{complete intro to the MessageVortex model}

\chapter{Intro}

\chapter{Requirements for an Anonymizing Protocol\label{sec:genRequirements}}
In the following sections, we elaborate on the main characteristics of the anonymizing protocol. 

The primary goal of the protocol is to enable Freedom of speech, as defined in Article 19 of the International Covenant on Civil and Political Rights (ICCPR)\cite{iccpr}.
\begin{quote}
	everyone shall have the right to hold opinions without interference 
\end{quote}
and
\begin{quote}
	Everyone shall have the right to freedom of expression; this right shall include freedom to seek, receive and impart information and ideas of all kinds, regardless of frontiers, either orally, in writing or print, in the form of art, or through any other media of his choice.
\end{quote}

We imply that not all participants on the Internet share this value. As of September \th{1}, 2016 Countries such as China (signatory), Cuba (signatory), Qatar, Saudi Arabia, Singapore, United Arab Emirates, or Myanmar did not ratify the ICCPR. Other countries such as the United States or Russia did either put local laws in place superseding the ICCPR or made reservations rendering parts of it ineffective. We may, therefore, safely assume that freedom of speech is not given on the Internet, as at least countries explicitly supersede them.

Network packets may pass through any point of the world. A sender has no control over it. This lack of control is since every routing device decides on its own for the next hop. This decision may be based on static rules or influenced by third party nodes or circumstances (e.g., BGB, RIP, OSPF\ldots). It is furthermore not possible to detect what way has a packet taken. The standard network diagnostic tool \verb|tracert| respectively \verb|traceroute| returns a potential list of hops. This list is only correct under certain circumstances (e.g., a stable route for multiple packets or same routing decisions regardless of other properties than the source and destination address). Any Output of these tools may, therefore, not taken as a log of routing decisions. There is no possibility in standard IP routed networks to foresee a route for a packet, nor can it be measured, recorded, or predicted before, while, or after sending. 

As an example of the problems analyzing a packet route, we may look at \verb|traceroute|. According to the man page of traceroute, \verb|traceroute| uses UDP, TCP, or ICMP packets with a short TTL and analyses the IP of the peer sending a TIME\_EXCEEDED (message of the ICMP protocol). This information is then collected and shown as a route. This route may be completely wrong. The man page describes some of the possible causes.

We cannot state that data packets we are sending are passing only through countries accepting the ICCPR to the full extent, nor can we craft packages following such a rule.

\begin{figure}[H]
	\begin{lstlisting}[language=bash,breaklines=true,basicstyle=\tiny]
	$ traceroute www.ietf.org
	traceroute to www.ietf.org.cdn.cloudflare-dnssec.net (104.20.0.85), 64 hops max
	1   147.86.8.253  0.418ms  0.593ms  0.421ms
	2   10.19.0.253  1.177ms  0.829ms  0.782ms
	3   10.19.0.253  0.620ms  0.427ms  0.402ms
	4   193.73.125.35  1.121ms  0.828ms  0.905ms
	5   193.73.125.81  2.991ms  2.450ms  2.414ms
	6   193.73.125.81  2.264ms  1.961ms  1.959ms
	7   192.43.192.196  6.472ms  199.543ms  201.152ms
	8   130.59.37.105  3.465ms  3.138ms  3.121ms
	9   130.59.36.34  3.904ms  3.897ms  4.989ms
	10   130.59.38.110  3.625ms  3.333ms  3.379ms
	11   130.59.36.93  7.518ms  7.232ms  7.246ms
	12   130.59.38.82  7.155ms  17.166ms  7.034ms
	13   80.249.211.140  22.749ms  22.415ms  22.467ms
	14   104.20.0.85  22.398ms  22.222ms  22.146ms
	\end{lstlisting}
	\caption{A traceroute to the host www.ietf.org}
\end{figure}

To enable freedom of speech, we need a mean of transport for messages which keep sender and recipient anonymous.

\section{Threat model\label{sec:adversary}}
We refer to jurisdiction as a geographical area where a set of legal rules created by a single actor or a group of actors apply, which contains executive capabilities (e.g., police, army, or secret service) to enforce this set of legal rules.

We assume for our protocol that adversaries are state-sponsored actors or players of large organizations. These actors have high funding and expected to have elaborated capabilities themselves or within reach of the sponsor. Actors may join forces with other actors as allies. However, achieving more than 50\% on a world scale is excluded from our model. We always assume one or more actors with disjoint interests covering half of the network or more. 

We assume the following goals for an adversary:
\begin{itemize}
	\item An adversary may want to disrupt non-authorized communication.
	\item An adversary may want to read any information passing through portions of the Internet.
	\item An adversary may want to build and conserve information about individuals or groups of individuals of any aspect of their life. 
\end{itemize}

To achieve these goals, we assume the following properties of our adversary:
\begin{itemize}
	\item An adversary has elaborated technical know-how to attack any infrastructure. This attack may cover any attack favoring his goals, starting with exploiting weaknesses of popular software (e.g., buffer overflows or zero-day exploits) down to simple or elaborated (D)DoS attacks.
	\item An adversary may monitor traffic at any point in public networks within a jurisdiction.
	\item An adversary may modify routing information within a jurisdiction freely.
	\item An adversary may freely modify even cryptographically weak secured data where a single or a limited number of entities grant proof of authenticity or privacy.
	\item An adversary may inject or modify any data on the network of a jurisdiction.
	\item An adversary may create their nodes in a network. He may furthermore monitor their behavior and data flow without limitation.
	\item An adversary may force a limited number of other non-allied nodes to expose their data to him. For this assumption, we explicitly excluded actors with disjoint interests.
	\item An adversary may have similar access to resources as within its jurisdiction in a limited number of other jurisdictions.
\end{itemize}

we may furthermore subdivide the adversaries into the following sub-classes:
\begin{itemize}
	\item A censoring adversary\\
	The primary goal of this adversary is censoring messages and opinions, not within his interests. He does this, regardless of whether the activities of censorship may be observed or not. Therefore, this adversary does not cloak its activities and typically bans censorship circumventing activities as illegal.
	\item An observing adversary\\
	This adversary behaves like a traditional spy. He collects and classifies information while hiding its activities. Unlike within reach of a censoring adversary, in this case, typically, no restrictions apply to the use of anonymization technology.
\end{itemize}

\section{Required Properties of an unobservable network}
In this section, we summarize the required properties of an anonymizing system.

\subsection{Anonymizing and Unlinking}
As we are unable to limit the route of our packets through named jurisdictions, we must protect ourselves from unintentionally breaking the law of a foreign country. Therefore, we need to be anonymous when sending or receiving messages. Unfortunately, most transport protocols (in fact, almost all of them such as \defref{SMTP}, SMS, \defref{XMPP}, or IP) use a globally unique identifier for senders and receivers, which are readable by any party which is capable of reading the packets. 

As a result, the anonymization of a sender or a receiver is not simple. A relay may allow at least the anonymization of the original sender given trust into the proxy. By combining it with encryption, we may even achieve a simple form of a sender and receiver pseudonymity. If cascading more relay like infrastructures and combining it with cryptography, we may achieve sender and receiver anonymity. When introducing anonymous remailing endpoints, we may additionally achieve both simultaneously.

These are the standard approaches in remailers and mixes. Their approaches are questionable as shown in \ref{sec:remailer} and \ref{sec:mixnets}. We have seen attacks on such systems in the past. Some of them were successful.

\subsection{Censorship Resistant}
In our scenario in \ref{sec:genRequirements}, we defined the adversary as someone with superior access to the network and its infrastructure. Such an adversary might attack a message flow in several ways:
\begin{itemize}
	\item Identify sender
	\item Identify recipient
	\item Read messages passed or extract meta information
	\item Disrupt communication fully or partially
\end{itemize}

We furthermore have to assume that all actions taken by a potential adversary are not subject to legal prosecution. This assumption based on the fact that an adversary trying to establish censorship may be part of the government of jurisdiction. We may safely assume that there are legal exceptions in some jurisdictions for such entities.

To be able to withstand an adversary outlined above, the messages sent requires to be unidentifiable by attributes or content. ``Attributes'' include any meta information including, but not limited to, frequency, timing, message size, sender, protocol, ports, or recipient.

\subsection{Controllable trust}
We have multiple options for relying on trust when building our system. We may rely on trust in infrastructure, we may work with distrust in infrastructure. In our model, we will work with distrust into the infrastructure. As every infrastructure node learns from each transaction (e.g., the usage of the network or size of messages), we have to minimize or ideally eradicate such information gains. A main problem is that we are unable to hide peer senders or recipients when routing messages. In jurisdictions where such infrastructure usage is illegal, we need to hide the presence of our routing messages from any party not trusted. Such hiding concludes that we need to be able to control which nodes are involved when sending messages. We refer to this concept as controllable trust.

In terms of trust, we have to conclude that:
\begin{enumerate}
	\item We trust in infrastructure because it is under full control of either the sender or the recipient.
	\item We should not trust all other infrastructure as an adversary is potentially able to misuse data passed through it.
\end{enumerate}

In this work, we work with both cases. We will, however, avoid whenever possible to trust in any third party apart from the sender and recipient.

\subsection{Reliable}
Any message-sending protocol needs to be reliable in its functionality. If the means of message transport are unreliable, users tend to use different means for communication\cite{zhou2011examining}. 

\subsection{Diagnoseable}
Transparent behavior is a prerequisite for reliability. If something is generating a  behavior, but we are unable to determine the reason for it (i.e., if we are expecting a different behavior), we usually assume a malfunction. Therefore ``reliable'' means not only stable by its behavior. It also means diagnoseable. A user's perception will not be ``reliable'' if he is not able to determine causes for differences in observed and expected behavior (e.g., \cite{nicholson2003assessing}).

\subsection{Available}
Availability has two meanings in this context, which do differ. Technology is available if\ldots
\begin{enumerate}
	\item a sender and a recipient have (or may have) the means of using it.
	\item the infrastructure provides the service (as opposed to: ``is running in a degraded or faulty state and, therefore, unable to provide the service'').
\end{enumerate}

The first meaning tells us that a protocol must run on infrastructure on which the user has access to it.

The second meaning tells us that messages must always be capable of flowing from the sender to the recipient. As a part of the infrastructure may fail at any time, the protocol must offer the possibility to send messages through alternate routes. Alternative routes are simple to achieve, and many protocols implement such redundancies already. However, taking into account that the sender and recipient are not known to a routing node, this is a goal hard to achieve. If we leave the choice of routing to any node apart from a trusted node, we will enable untrusted nodes to manipulate routing decisions and thus affect the security of a message.

\subsection{Identifiable Sender}
A messaging system offering unlinkability may offer sender anonymity. If so, a sender should be identifiable in such a way, that a classification of senders is possible at any time, and impersonation is not achievable. It is important to understand that an identifiable sender does not necessarily mean that we can identify a sender as a specific party. In our case, any identification will do, which offers non-hijackable pseudonymity. We decided to go for a short-lived pseudonymity (see \defref{eID} in section \ref{sec:ephemeralIdentity}). This system guarantees that while only a pseudonym of the sender is known, the hijacking of data by other participants of the system is not possible.


\chapter{Rationale}


\chapter{Protocol Outline}
\fxwarning{complete section}
\chapter{Key Components}
\fxwarning{complete section}
\section{Nodes}
\fxwarning{complete section}
\section{Protocol Layers}
\fxwarning{complete section}
\section{Vortex Messages}
\fxwarning{complete section}
\section{Workspaces}
\fxwarning{complete section}
\section{Ephemeral Identities}
\fxwarning{complete section}
\section{Routing Operations}
\fxwarning{complete section}
\section{Routing}
\fxwarning{complete section}



\chapter{Transport Layer and Message Blending}
\fxwarning{complete section}
\section{Plain Blending}
\fxwarning{complete section}
\section{F5 Blending}
\fxwarning{complete section}

\chapter{Message Structure}
\fxwarning{complete section}
\section{Identification of a Message}
\fxwarning{complete section}
\section{Message Structure Related to Censorship Circumvention}
\fxwarning{complete section}
\section{Message Structure Related to Information Leaking}
\fxwarning{complete section}

