%!TeX program=pdflatex
%!TeX encoding=utf8
%!TeX spellcheck = en_US
%!TeX root = ../../messageVortex.tex
\partepigraph{Limit your inputs to only those that support a certain kind of self-destructive behavior, and you can be cheered with enthusiasm as you drive yourself off a cliff.}{Adam-Troy Castro}
\part{Discussion on Results}
\fxwarning{complete section}

\chapter{Measuring up to the Requirenets\label{sec:reqDiscussion}}
In this section we analyze the level of achievment in respect to the requirements defined in section\ref{sec:requirements}. We will go through each requirement and discuss the level of achievement. In the case of a failure, we highlight the reason for the failure and elaborate on the consequences of the current flaws.

\fxwarning{complete section}


\chapter{Achieved level of anonymity}
\fxwarning{complete section}

\chapter{Weaknesses of the protocol}
\fxwarning{complete section}

\chapter{Further and Missing Research}
The current blending layer is simple in its inner working. It creates context-less messages based on an easily recognizable scheme. An unsuspecting observer may have the impression that this is just a way of communicating, but censor may, by observing the message flow easily and conclude that these messages are not written by a human. Such detection could lead to censorship of the respective routing node and thus disrupt the message flow. It is easy to recover from such censorship by advertising a new identity to known peer partners. To minimize the effects of censorship, an improvement in this area would help.

To be undetectable, all work done by the blending layer has to be indistinguishable from regular human communication. This applies not only to the message steganographic embedding of the message but to the message content as well. This is very much similar to the problems of chatterbots these days. Assuming that a blending layer is only communicating with other nodes correctly embedding messages, we have a chatterbot problem. It is reduced as the chatterbot must only reply credibly and undetectable to generated messages of other chatterbots. If assuming that a blending layer replies to any non-Vortex nodes, the problem boils down to a Turing test, as stated in \cite{turing1950computing}. As we defined that an adversary has enormous but limited resources, this blending is, however, sufficient if it is done ``good enough''. What criteria would apply here is a topic for further research. Applying any research to this topic would require to add a more precise adversary model.

The currently applied choice of transport layer protocol is a snapshot of current Internet traffic. While done with great care, it must be adapted to the changing communication habits of humanity. Identifying new or depreciated communication protocols and blending schemes would be another field of research.

A comprehensive survey of the newest trends and techniques in steganography is another topic to be covered. It would allow identifying new candidates for blending techniques. Especially interesting are steganography algorithms covering movie file formats.

This is especially hard since true evidence of in-depth protocol usage seems to be completely missing. While we were able to gather much data which is collected by simple routers (such as bandwidth), credible figures about client and content usage seemed to be completely missing or of very poor quality.

Anonymity has effects on the behavior of humans. We have found that although there is some research in this field (such as \cite{postmes2001social}), the evidence is very weak. Although the possibility of anonymity is undisputed among so-called free countries, the downsides (e.g., misuse for criminal acts) of anonymity are apparent. More research in this field is required. On the other hand, a lack of awareness for anonymity, especially in ``non-free'' jurisdiction, has been observed, which would be another relevant field of research. 

\fxwarning{Mention that writing simulators is not the same as writing an implementation. Mention that steganography.}
\fxwarning{complete section}


