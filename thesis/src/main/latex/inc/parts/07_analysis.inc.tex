%!TeX program=pdflatex
%!TeX encoding=utf8
%!TeX spellcheck = en_US
%!TeX root = ../../messageVortex.tex
\partepigraph{Atoms are very special: they like certain particular partners, certain particular directions, and so on. It is the job of physics to analyze why each one wants what it wants.}{Richard P. Feynman}
\part{Analysis of MessageVortex\label{sec:analysis}}
In section~\ref{sec:adversary}, we described two different kinds of adversaries. Which do require different properties for our system to be fulfilled.

An observing adversary is the less dangerous one. While observing all traffic he may not disrupt communication. Instead he is using all available information to collect data about all items of intrest (\defref{IoI}). He may do this for example by colecting inside or outside information about all message flows he may encounter. He may use this information and assign it to certain individuals or groups of individuals.

A censoring attacker is far more dangerous as he does not only observe the system, but he may systematically supress freedom of speech and all technology related to it. As he has the means and the technical know-how, he may try, apart from observing, to discover systems communicating illegally either by observation or by infiltration of systemms. He may furthermore track down individuals within its reach and prosecute them. All other participants of an illegal system may be either identified and blacklisted or even attacked either by infiltrating their systems or by effectively launching DoS attacks against those systems.

In the following sections, we will analyze aspects of security and highlight differences in terms of the different adversaries.



\chapter{Analysis of the Blending Layer}
\fxwarning{complete section Analysis blending layer}

\chapter{Analysis of the Core Operations}
\fxwarning{complete section analysis of operations}

\section{Splitting and Merging}
\fxwarning{complete section Analysis of Splitting OP}

\section{Encryption and Decryption Operations}
\fxwarning{complete section Analysis of Encryption OP}

\section{Add and Remove Redundancy Operations}\label{sec:analysisReedSolomon}
\fxwarning{complete section Analysis of Redundancy Op}

%%%%%%%%%%%%%%%%%%%%%%%%%%%%%%%
%%% Preplaced float
%%%%%%%%%%%%%%%%%%%%%%%%%%%%%%%
\begin{figure*}[!t]\centering
	\includegraphics[width=1\textwidth]{inc/randomblock_10kb}
	\caption{Resulting entropy of addRedundancy with and without encryption step}
	\label{fig:entropy}
\end{figure*}


\fxwarning{Fill in results of the RS analysis regarding the findings of the protocol}

\fxwarning{FIXME story:
	Argue that using reed solomon operations instead of plain Lagrange allows detection of cheaters with well known algorithms
}

\chapter{Well Known Attacks}\label{sec:wellKnownAttacks}
In the following sections, we emphasize on possible attacks to an anonymity preserving protocols. These attacks may be used to attack the anonymity of any entity involved in the message channel. In a later stage, we test the protocol for immunity against these classes of attacks.

\section{Broken Encryption Algorithms}
Encryption algorithms may become broken at any time. This either to new findings in attacking them, by more resources being available to an adversary, or by new technologies allowing new kinds of attacks. A proper protocol must be able to react to such threads promptly. This reaction should not rely on a required update of the infrastructure. Users should solely control the grade of security. 

We cannot do a lot for attacks of this kind to happen. However, we might introduce a choice of algorithms, paddings, modes, and key sizes to give the user a choice in the degree of security he wants to have.

We have introduced a way to support a set of independent cryptographic algorithms, paddings ,modes, and prngs. The support of these algorithms does not have to be uniform throughout the system, instead it is sufficient for two neighbouring nodes support the same algoithms in order to be used. 

\section{Attacks Targeting Anonymity}
Attacks targeting users anonymity are the main focus of this work. Many pieces of information may be leaked, and the primary goal should, therefore, rely on the principles established in security.

\begin{itemize}
	\item Prevent an attack\\
	Attack prevention can only be done for attacks that are already known and may not be realistic in all cases. In our protocol, we have strict boundaries defined. A node under attack should at any time of protocol usage (this excepts bandwidth depletion attacks) be able to block malicious identities. Since establishing new identities is costly for an attacker, he should always require far more resources than the defender.
	\item Minimize attack surface\\
	This part of the attack prevention is included by design in the protocol.
	\item Redirect an attack\\
	Although the implementation does not do this, it is possible to handle suspected malicious nodes differently.
	\item Control damage\\
	For us, this means leaving as little information about identities or meta information as possible on untrusted infrastructures. If we leave traces (i.e., message flows, or accounting information) they should have the least possible information content and should expire within a reasonable amount of time.
	\item Discover an attack\\
	The protocol is designed in such a way that attack discovery (such as a query attack) is possible. However, we consider active attacks just as part of the regular message flow. The protocol must mitigate such attacks by design.
	\item Recover from an attack\\
	An attack does always impose a load onto a system's resources regardless of its success. It is vital that a system recovers almost immediately from an attack and is not covered in a non-functional or only partial-functional state either temporarily or permanently.
\end{itemize}

In the following subsections, we list a couple of attack classes that have been used against systems listed in \ref{sec:implSystems} or the respective academic works. We list the countermeasures which have been taken to deflect these attacks.

\subsubsection{Probing Attacks}
Identifying a node by probing and check their reaction is commonly done when fingerprinting a service. As a node is participating in a network and relaying messages probing may not be evaded. However, it may be made costly for an adversary to do systematic probing. This should be taken into account. Both currently specified transport protocol features an indefinite number of possible accounts. Since not the server but the endpoint address is behaving, node probing is more complicated than in other cases where probing of service is sufficient. 

One of the problems is clear-text requests. These requests may be used on any transport layer account without previous knowledge of any host key. Thus the recommendation in table \ref{tab:protoReplyCrit} is generally not to answer the requests. Routing nodes in jurisdictions not fearing legal repression or prosecution may reply to clear text requests, but it is usually discouraged as they allow harvesting of addresses.

One strategy to avoid would be to put high costs onto clear-text requests in such a way that a clear-text request may have a long reply time (e.g., up to one day). A node is free to blacklist an identity in case of an early reply. This is an insufficient strategy as a big adversary may have lots of identities in stock. Requesting an unusually long key as a plain-text identity does not make sense either as these as well may be kept in stock. We may, however, force a plaintext request to have an identity block with a hash following specific rules. We may, for example, put in a requirement that the first four bytes of the hash of a header block translates to the first four characters of the routing block. At the moment, this has been rejected in the standard for practical reasons. First, as the request is unsolicited, a sender is the only one able to decide the algorithm of the hash. This would allow a requester to choose upon the complexity of the puzzle. Second, any negotiation of the cost of the request would result in the disclosure of the node as VortexNode, which might be unsuitable.

\subsubsection{Hotspot Attacks}
Hotspot attacks aim to isolate high traffic sites within a network. By analyzing specific properties or the general throughput locations with outstanding traffic may be identified. These messages do quite often reveal senders or recipients. Sometimes an intermediate node in an anonymizing system. 

\subsubsection{Message Tagging and Tracing}
When using an anonymization system, a message may be either fully or partially traced or even tagged. Tagging allows one to recognize a message at a later stage and map it to its predecessors. Protocols with tagable messages are not suitable for anonymization systems.

\subsubsection{Side Channel Attacks}
Side-channel attacks are numerous. Especially important to us are attacks related to either lookup in independent channels (e.g., downloading of auxiliary content of a message) or behavior related to timing patterns.

\subsubsection{Sizing Attacks}
There are two kinds of sizing attacks identified to be relevant for us. One is the possibility for matching messages with related sizes, and the other one is to relate message size to the original messages. Both attacks may be considered as a tracing attack and will be analyzed accordingly.

\subsubsection{Bugging Attacks}
Numerous attacks are available through the bugging of a protocol. In this chapter, we outline some of the possibilities and how they may be countered:

\begin{itemize}
	\item Bugging through certificate or identity lookup:\\
	Almost all kinds of proof of identities, such as certificates, offer some revocation facility. While this is a perfect desirable property of these infrastructures, they offer a flaw. Since the location of this revocation information is typically embedded in the proof of identity, an evil attacker might use a falsified proof of identity with a recording revocation point.
	
	There are multiple possibilities to counter such an attack. The easiest one is to do no verification at all. Having no verification is, however, not desirable from the security point of view. Another possibility is only to verify trusted proof of identities. By doing so, the only attacker could be someone having access to a trusted source of proof of identities. A third possibility is to relay the request to another host either by using an anonymity structure such as Tor or by using its infrastructure. Using Tor would violate the ``Zero Trust'' goal. Such a measure would only conceal the source of the verification. It would not hide the fact that the message is processed. A fourth and most promising technology would be to force the sender of the certificate to include a ``proof of non-revocation''. Such a proof could be a timestamped and signed partial CRL. It would allow a node to verify the validity of a certificate without being forced to disclose itself by doing a verification. On the downside has to be mentioned that including proof of non-revocation involves the requirement to accept a certain amount of caching time to be accepted. This allowed caching time reduces the value of the proof as it may be expired in the meantime. It is recommended to keep the maximum cache time as low as 1d to avoid that revoked certificates may be used. 
	
	\item Bugging through DNS traffic:\\
	A standard protocol on the Internet is DNS. Almost all network-related programs use it without thinking. Typically the use of such protocol is only a minor issue since the resolution of a lookup usually done by an ISP. In the case of a small Internet service provider (ISP), this might, however, already become a problem.
	
	The bugging in general attack works as follows: We include a unique DNS name to be resolved by a recipient. This can be done most easily by adding an external resource such as an image. A recipient will process this resource and might, therefore, deliver information about the frequency of reading, or the type of client. 
	
	It must be taken into account that the transport layer will always do DNS lookups and that we may not avoid this attack completely. We may, however, minimize the possibilities of this attack.
	
	\item Bugging through external resources:\\
	A straightforward attack is always to include external resources into a message and wait until they are fetched. In order to avoid this kind of attack, plain text or other self-contained formats should be used when sending a message. As we may not govern the type of contained message, we can make at least recommendations concerning its structure.
\end{itemize}

\subsubsection{Analysis by Building Interaction Graphs\label{sec:analysisInteractionGraphs}}
\fxwarning{Write section Analysis on Graphs}

\subsection{Denial of Service Attacks}
\subsubsection{Censorship}
Whereas traditional censorship is widely regarded as selective information filtering and alteration, very repressive censorship can even include denial of information flows in general. Any anonymity system not offering the possibility to hide in legitimate information flows, therefore not censorship-resistant.

\subsubsection{Denial of service}
An adversary may flood the system in two ways.
\begin{itemize}
	\item He may flood the transport layer exhausting resources of the transport system.\\
	This is a straightforward attack. MessageVortex has no control over the existing transport protocol. Therefore, all flooding attacks on that layer are still effective. However, If an adversary attacks a node, the redundancy of a message may still be sufficient. On the other hand, flooding disrupts at least all other services using the same transport layer on that node. This result may be unacceptable for an attacker. More likely would be censorship.
	\item He may flood the routing layer with invalid messages.\\ 
	Identifying the messages is relatively easy for a node. Usually, it should be sufficient to decode the CPREFIX block of a message. If the CPREFIX is valid, then the header block either identifies a valid identity or processing may be aborted. 
	\item He may flood an accounting layer with newIdentity.\\
	Flooding an accounting layer with identities is possible. Since the accounting layer is capable of adapting costs to a new identity, it may counter this attack by giving large puzzles to new identities. This affects all new identities and not only those flooding. If a flooding attack is carried out over a long time, a node may decide to split its identity. All recent active users get a new identity, whereas the old one opposes high costs. This would force an attacker to work in intervals and is no longer able to make a permanent DoS attack.
\end{itemize}

\subsubsection{Credibility Attack}
Another type of DoS attack is the credibility attack. While not a technical attack, it is very effective. A system not having a sufficiently big user base is offering thus a lousy level of anonymity because the anonymity set is too small or the traffic concealing message flow is insufficient. 

Another way is to attack the reputation of a system in such a way that the system is no longer used. An adversary has many options to achieve such a reduction in credibility. Examples:
\begin{itemize}
	\item Disrupt functionality of a system.\\ 
	This may be done by blocking of the messaging protocol it uses or by blocking messages. Furthermore, an adversary reduces functionality when removing known participants from the network either by law or by threatening.
	\item Publicly dispute the effectiveness of a system.\\
	Disputing the effectiveness is a very effective way to destroy a system. People are not willing to use a system which believed to be compromised if the primary goal of using the system is avoiding being observed.
	\item Reduce the effectiveness of a system.\\
	A system may be considerably loaded by an adversary to decrease the positive reception of the system. He may further use the system to send \defref{UBM} to reduce the overall experience when using the system. Another way of reducing effectiveness is to misuse the system for evil purposes such as blackmailing and making them public.
	\item Dispute the credibility of the system founders.\\
	Another way of reducing the credibility of a system is to undermine its creators. If -- for example -- people believe that a founders' interest was to create a honey pot (e.g., because he is working for a potential state-sponsored adversary) for personal secrets, they will not be willing to use it.
	\item Dispute the credibility of the infrastructure.\\
	If the infrastructure is known or suspected to be run by a potential adversary, people's willingness to believe in such a system is expected to be drastically reduced.
\end{itemize}


\chapter{Identification of Possible Attack Schemes and Mitigation\label{sec:attacks}}
In this chapter we take the attacks identified in the section~\ref{sec:wellKnownAttacks} and analyze our protocol on whether it is susceptible on such attacks or not.

\section{Static Attacks}
\fxwarning{complete section}

\subsection{Bugging and Tagging Attacks}
\fxwarning{complete section}

\subsection{Information Leaking Related to Information Available to Routing Nodes}
\fxwarning{complete section}

\subsection{Identification of involved \VortexNodes{}}
\fxwarning{complete section}

\subsection{Identification of \MessageVortex{} Traffic}
\fxwarning{complete section}

\section{Dynamic Attacks}
\fxwarning{complete section}

\subsection{Attacks against the \MessageVortex{} system itself}
\fxwarning{complete sction}

\subsubsection{DoS Attacks against the Transport System}
\fxwarning{complete sction}

\subsubsection{DoS by Traffic Replay}
\fxwarning{complete sction}

\subsubsection{DoS by Traffic generation}
\fxwarning{complete sction}

\subsection{Attacking a single ephemeral Identity of a MessageVortex Node}
\fxwarning{complete section}

\subsubsection{Denial of Service by Exhausting Quotas or Limits}
\fxwarning{complete sction}

\subsection{Attacking Sending and Receiving Identities of the MessageVortex System}
\fxwarning{complete section}

\subsubsection{Traffic Highlighting or Traffic Analysis}
\fxwarning{complete sction}

\subsection{Recovery of Previously Carried Out Operations}
\fxwarning{complete section}

\chapter{Analysis of the effectiveness of Attack Schemes}
\fxwarning{complete section}

\chapter{Analysis of the Degree of Anonymization in Comparison to other Systems}
\fxwarning{complete section}

\chapter{To be placed (TBP)}
\section{Side Channel Leaking}
\fxwarning{complete sction}

\subsection{Software Updates and Related Data Streams}
\fxwarning{complete sction}

\subsection{Bugging in transported messages}
\fxwarning{complete sction}
