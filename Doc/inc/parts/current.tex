\chapter{Current situation}
As of today the de facto standard for asynchronous mail transfer is SMTP as defined in RFC5321\cite{RFC5321} and its predecessors. While the transfer protocol SMTP is quite compact, the protocol is enhanced with several standards for encryption, multimedia support and similar. A mail client offers today various support for a lot of sub-protocols. The following list is an excerpt of related sub-protocols which are either related to transport, reliability, identification or encryption. 

\section{Implemented protocols}
\subsection{SMTP \index{SMTP|textbf}}
The SMTP protocol is currently specified in \cite{RFC5321}. It specifies a method to deliver reliably asynchronous messages thru a specific transport medium (most of the time the internet). The Protocol makes a dstinction between a mail envelope and its content. The envelope contains the routing information which is the sender and the recipient. The content again is split into two parts. These parts are the headers (which do contain meta information about the message such as subject, reply address or a comprehensive list of all recipients) and the body which contains the message itself.\par

It furthermore introduces a simplistic model for mail communication. A more comprehensive model is introduced in the section \nameref{sec:mailTransport}. As the proposed model is not sufficient for a comprehensive end-to-end analysis.\par



FIXME incomplete section

\subsubsection{Mail transport\index{Message!Transport|textbf}}
\cite{RFC1870}
FIXME incomplete section

\subsubsection{encryption}
Encryption is the only anonymizing technology which is available. There are serveral kind of encryptions which have to be differentiated. Link encryption controls the E-Mail connection a guarantees that the whole communication between two servers is encrypted. It does however not guarantee that the message and routing information is protected all the way thru the network. Message encryption is a weaker encryption which is done at a higher level of the protocol stack. It guarantees that a message is end to end encrypted but discloses all routing and header information.\par

One kind of Mail link encryption is specified in \cite{RFC3207}. This RFC specifies that when a STARTTLS-Command is issued a TLS handshake initiating a encrypted link should be carried out between two Servers. Only not public servers (not published in DNS using MX records) may enforce the use of TLS. All public servers must allow non-TLS transport. Authentication thru this port is possible but usually not done. The STARTTLS specification states clearly that securing a link provides no end-to-end security. An attack to this mechanism is very simple. The only thing required is injecting a 454 error code when the client issues a STARTTLS. According to the document the sending server may then refuse to deliver the document but in reality this never happens in public SMTP servers.\par

For encryption between a mail endpoint (repective its MSA)  and the server MRA Clients may choose to use alternate ports which enforce a TLS handshake at the TCP handshake. This invalidates the possibility to disturb a connection while still in plain text modes with fake errorcodes but since it is a weak security anyway it makes really a difference. According to the \cite{RFC3207} document the port 587 should be used. On some servers the same functionality is provided on port 465. This was originally intended for mail transmission between two MSAs. The usage of this port has however never been standardized, violates \cite{RFC6409} and the port has been assigned to the URD Protocol by IANA.\par

The second type of encryption is message encryption. Message encryption does not cover the whole server communication starting from a specific point. It does only cover some parts or the full message body. The Two main protocols in use are S/MIME (As specified in \cite{RFC2311}) and PGP/MIME (as specified in \cite {RFC2015}; bases on \cite{RFC1847}). Both do reveal vital information to all involved parties of the mail transport and a possible third party observer thus completely invalidating anonymity. Informations which are visible to anyone are: 

\begin{itemize}
\item sender address (may be forged)
\item sender client (may be forged)
\item Recipient address
\item message subject 
\item the full routing path including all rewrites, timing information and intermediate hops.
\item the content type
\item Mime-Version
\item Date and time of sending
\end{itemize}

Any client or intermediate Server may furthermore add additional information of any kind (such as virus scanning information, anti spam taxation, reply address).

FIXME unfinished section

\subsection{MIME}
\cite{RFC2045}
\cite{RFC2046}
\cite{RFC2047}
\cite{RFC2048}
\cite{RFC2049}
FIXME incomplete section

\subsubsection{S/MIME}
\cite{RFC3851}
FIXME incomplete section

\subsubsection{PGP/MIME}
\cite{RFC2440}
FIXME incomplete section

\subsection{DNS}
\cite{RFC2929}
FIXME incomplete section

\subsubsection{DNSSEC}
\cite{RFC5155}
FIXME incomplete section

\subsubsection{Sender Policy Framework}
\cite{RFC4408}
\cite{RFC6652}
FIXME incomplete section

\subsubsection{Sender ID}
\cite{RFC4401}
FIXME incomplete section

\subsection{Transport Protocols}
FIXME incomplete section

\subsubsection{IPv4}
\cite{RFC760}
\cite{RFC791}
\cite{RFC1180}
\cite{RFC792}
\cite{RFC919}
\cite{RFC922}
\cite{RFC950}
\cite[p.~3]{RFC793}
FIXME incomplete section

\subsubsection{IPv6}
\cite{RFC2460}
FIXME incomplete section

\subsubsection{TCP}
FIXME incomplete section

\subsection{Remote MDA protocols}
FIXME incomplete section

\subsubsection{POP3}
\cite{RFC1939}
FIXME incomplete section

\subsubsection{IMAP}
\cite{RFC3501}
FIXME incomplete section

