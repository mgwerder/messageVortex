\chapter{Analysis of current situation}\label{chap:analysis}
FIXME waiting for this text to appear

\section{Current state of common Technology}
FIXME incomplete section

\subsection{Mailrouting}
FIXME incomplete section

\subsubsection{SMTP}
FIXME incomplete section

\subsubsection{LMTP}
FIXME incomplete section

\subsubsection{IMAP}
FIXME incomplete section

\subsubsection{POP}
FIXME incomplete section

\subsubsection{MS-OXMAPIHTTP}
FIXME incomplete section

\section{Current state of available Technology}
FIXME incomplete section

\section{Missing Gap}
FIXME incomplete section

\section{Skeleton of Mails and mail transfer}\label{sec:mailTransportSkeleton}
As shown in figure \ref{fig:MailAgents} on page \pageref{fig:MailAgents} a typical mail starts with its creation in a mail client. The message is sent by SMTP\defref{SMTP} to the providers ISP which accepts the mail for relaying if the user is identifiable. As identification an IP or a username/password tupel is typically accepted. The mail is stored in a local queue and the final recipient server is determined by the domain name of the recipient mail address.\par

The next hop of the mail is determined by looking up the MX record of the recipient domain. If multiple MX records are found then the server with the lowest "`preference number"' is taken and an SMTP daemon tries to send the mail to that host usin SMTP. Unlike the first SMTP server this server will accept the mail without prior authentication as he relays mail for the recipient domain.\par

The receiving mailserver may apply secondary analysis to the mail (such as anti virus or anti spam analysis) and then pass it to the mailbox of the final recipient.\par

A recipient may connect at any time to his mailbox thru the providers \defref{Remote MDA} (Typically using IMAP, pop, or MS-OXMAPIHTTP) and pickup the mail.\par

The scenario described above is a very simple case of mail delivery and ignores a lot of possible scenarios. All of these Scenarios are usually intermediate steps covered by the SMTP protocol.\par


